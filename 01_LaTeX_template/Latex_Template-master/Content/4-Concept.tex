%% ==============================
\chapter{\iflanguage{ngerman}{Konzept}{Technogical Fundations}}
\label{sec:Technogical Fundations}
%% ==============================
This chapter introduces technological foundations fundamental for this thesis, which includes methods, models and tools that are used later in the thesis. The first section focus on the basic principle of Unity3D, which is used to build simulation environment for generating synthetic images. Then the second section shortly explain the essential knowledge of deep learning, especially convolution neural network(CNN), which used by modeling to recognize objects and predict the 3D-pose of them. Section 2.2 give an overview of the Kreas/Tensorflow framework, which will be used for training the CNN-model. Section 2.3 shows how to build a shallow neural network model in Keras and train it.

\section{Unity3D background}
Unity is a cross-platform game engine developed by Unity Technologies. Due to the motivation of transfer learning from simulation to real environment, a simulation scene in virtual environment need to be built and large numbers of synthetic image data of it are requisite. The unity engine gives users the ability to create virtual objects in 3D and offers a primary scripting API in C\# to render images that are needed in this research. All these features shows that Unity is suitable for rendering requests and can simplify the works during modeling. 
\subsection{User interface}
Unity provides a basic graphical interface to build the simulation environment, which includes five basic windows. 
\begin{itemize}
	\item The Scene View: where user work with game objects, including models, lights and colliders, to construct the user's scenes. 
	\item The Game View: where user can preview and play simulation scene as a work in progress as user develop it.
	\item The hierarchy window: All of the game objects in the open scenes are listed by the hierarchy window in hierarchical order.
	\item The project window: where user imports, stores and edits the Asset files. 
	\item The inspector window: It is context sensitive and displays all the properties of any selected game object, asset or setting.
\end{itemize} 
All the modeling works can be done in the graphical user interface. A general method of modeling are divided into 4 steps. Firstly a new default model needs to be created in Unity or imported from other 3D modeling software. This new model will appear in hierarchy window with other existent objects. Here the hierarchical relations can be adjusted. Secondly the dimensions and positions of new created model can be modified in inspector window, and new components (e.g. C\# scripts, Mesh Renderer, etc.) can be added into the model. Then in scene window a arbitrary perspective could be set, and according to the local coordinate system the position relationship among different objects can be adjusted. Finally when all items are correctly set and all needed components are added on them, the simulation scene can be played and the progress can be monitored in game view window in real time.
\subsection{Coordinate system in Unity}
The world coordinate system is left handed (as Direct X) where x positive axis is right, y positive is up and z is positive into the screen.\\
\begin{figure}[h]
	\centering
	\captionsetup{justification=centering}
	\includegraphics{{Figures/Section3_CoordinateSystem.png}}
	\caption{Coordinate system in Unity}
	\label{fig: Coordinate unity}
\end{figure}\\
Screen coordinate system is bottom-up: (0,0) at bottom-left corner and (pixelWidth-1,pixelHeight-1) at right-top; x axis is positive right and y is positive up. The z position is in world units from the camera.\\
Viewport coordinate system is normalized and relative to the camera, so the bottom-left point is (0,0), the top-right is (1,1). The z position is in world units from the camera.\\
Finally UI coordinate system is top-down: the y coordinate varies from zero at the top edge of the window to a maximum at the bottom edge of the window. The upper-left point is (0,0); the bottom-right is (1,1).\\
There are many functions to convert between all these different coordinate systems
\subsection{Basic concepts in scene}
To build a basic scene in Unity 3D, some important core concepts need to be known. which include Game Object, Component, Script and etc. They form the main basic elements of unity and should be added into the scene to realize the needed functions and effects. Normally the Component and Script are sub-element of the Game Object.\\
\begin{itemize}
	\item Scenes: Scenes contain the objects of simulation project. They can be used to create a main menu, individual levels, and anything else. Each unique Scene file is a unique level. In each Scene, the environments, obstacles, decorations will be placed, and the project are designed and built in pieces.\\
	A new Unity project will show a new Scene. The scene will be empty except for default objects- either an orthographic camera, or a  perspective camera and a directional light. 
	\item GameObjects: The GameObject is the most important concept in the Unity Editor.\\
	Every object in project is a GameObject. This means that everything which can be thought of to be in the project has to be a GameObject. However, a GameObject can't do anything on its own; The properties need to be given before it can become a character, an environment, or a special effect.\\
	A GameObject is a container; pieces are added to the GameObject container to make it into a character, a light, or whatever needed in project. Each added piece is called a component.\\
	Depending on what kind of object need to be created, different combinations of components can be added to a GameObject. A GameObject could be thought of as an empty cooking pot, and components are different ingredients that make up the recipe of project.  
	\item Component: Components are the nuts \& bolts of objects and behaviors in a game. They are the functional pieces of every GameObject.\\
	A GameObject is a container for many different Components. By default, all GameObjects automatically have a Transform Component. This is because the Transform dictates where the GameObject is located, and how it is rotated and scaled. Without a Transform Component, the GameObject wouldn't have a location in the word.
	\item Script: Scripting is an essential ingredient in all projects. They can be used to create graphical effects, control the physical behavior of objects and even implement a custom automatic system in the project. In next subsection, basic knowledge of scripting will be introduced to show how the script runs in unity and which basic functions are included in it. 
\end{itemize}
\subsection{Basic knowledge of script}
Unity supports the C\# programming language, which is an industry-standard language similar to Java or C++.\\
A script makes its connection with the internal workings of Unity by implementing a class which derives from the built-in class called MonoBehaviour. A class is like a kind of blueprint for creating a new Component type that can be attached to GameObjects. The core elements to realize different effects are the Event Functions. A script in Unity is not like the traditional idea of a program where the code runs continuously in a loop until it completes its task. Instead, Unity passes control to a script intermittently by calling certain functions that are declared within it. Once a function has finished executing, control is passed back to Unity. These functions are known as event functions since they are activated by Unity in response to events that occur during gameplay. Unity uses a naming scheme to identify which function to call for a particular event. There are four basic types of events:
\begin{itemize}
	\item Regular Update Events: 
	\begin{enumerate}
		\item Update function: The Update function is the main place for making changes to position, state and behavior of objects in the scene just before each frame is rendered. Update is called before the frame is rendered and also before animations are calculated.
		\item FixedUpdate function: The physics engine also update in discrete time steps in a similar way to the frame rendering. A separate event function called FixedUpdate is called just before each physics update.
		\item LateUpdate function: It is also useful sometimes to be able to make additional changes at a point after the Update and FixedUpdate functions have been called for all objects in the scene and fter all animations have been calculated.
	\end{enumerate}
	\item Initialization Events:
	\begin{enumerate}
		\item Start function: It is called before the first frame or physics update on an object.
		\item Awake function: The Awake function is called for each object in the scene at the time when the scene loads. All the Awakes will have finished before the first Start is called. This means that code in a Start function can make use of other initializations previously carried out in the Awake phase.
	\end{enumerate}
	\item GUI event: Unity has a system for rendering GUI controls over the main action in the scene and responding to clicks on these controls. This code is handled somewhat differently from the normal frame update and so it should be placed in the OnGUI function, which will be called periodically.
	\item Physics events: The physics engine will report collisions against an object by calling event functions on the object's script. 		
\end{itemize}
\begin{figure}[h]
	\centering
	\captionsetup{justification=centering}
	\includegraphics[width=\textwidth]{Figures/Section3_Eventfunction.png}
	\caption{Event function and execution order in Unity}
	\label{fig: event_function}
\end{figure}

\section{Neural network and Deep Learning}
In this section a introduction to neural network and deep learning are presented with the aim of enabling the reader to build up a precise understanding of the concepts and ideas as well as of the thesis' context.\\
The simplest definition of a neural network, more properly referred to as an 'artificial' neural network (ANN), is provided by the inventor of one of the first neuro-computers, Dr.Robert Hecht-Nielsen. "It is a computing system made up of a number of simple, highly interconnected processing elements, which process information by their dynamic state response to external inputs."\\
ANNs are processing devices (algorithms or actual hardware) that are highly simplified technical implementations for modeling information processing in the brain and nervous system.
\subsection{Biological motivation and connections}
The basic computational unit of the brain is a neuron. Approximately 86 billion neurons can be found in the human nervous system and they are connected with approximately $10^14$-$10^15$ synapses. The diagram below shows a cartoon drawing of a biological neuron and a common mathematical model.
\begin{figure}[h]
	
	\begin{subfigure}{0.5\textwidth}
		\includegraphics[width=0.9\linewidth, height=4cm]{Figures/Section3_NeuroStructure.png} 
		\captionsetup{justification=centering}
		\caption{Biological Neuron}
		\label{fig:Bio neuron}
	\end{subfigure}
	\begin{subfigure}{0.5\textwidth}
			\includegraphics[width=0.9\linewidth, height=4cm]{Figures/Section3_NeuroMathe.jpeg}
			\captionsetup{justification=centering}
			\caption{mathematical model of neuron}
			\label{fig:mathe neuron}
	\end{subfigure}
\captionsetup{justification=centering}
\caption{Basic Neuron Unit}
\label{fig:unit of a neuron}
\end{figure}

The idea is that the synaptic strengths (the weights $w$) are learnable and control the strength of influence and its direction: excitory (positive weight) or inhibitory (negative weight) of one neuron on another. In the basic model, the dendrites carry the signal to the cell body where they all get summed. If the final sum is above a certain threshold, the neuron can fire, sending a spike along its axon. In the computational model the weighted sum of inputs then passes through an \textbf{activation function}, whose general purpose is to model the "firing rate" of a biological neuron by converting the weighted sum into a new number according to a formula.
\subsection{Neural Network Architecture}
Deep learning is the name used for "stacked neural networks", that means networks composed of several layers. The layers are made of nodes. A node is just a place where computation happens, loosely patterned on a neuron in the human brain. 

\begin{enumerate}
	\item \textbf{Structure of an artificial neuron}\\
	The basic unit of computation in a neural network is the neuron, often called a node or unit. It receives input from some other nodes, or form an external source and computes an output. Each input has an associated weight (w), which is assigned on the basis of its relative importance to other inputs. The node applies a function to the weighted sum of its inputs. An example of the structure of a neural is shown in Figure 4.4.
	\begin{figure}[h]
		\begin{subfigure}{0.5\textwidth}
			\includegraphics[width=0.9\linewidth, height=4cm]{Figures/Section3_OneNeuron_1.png} 
		\end{subfigure}
		\begin{subfigure}{0.5\textwidth}
			\includegraphics[width=0.9\linewidth, height=4cm]{Figures/Section3_OneNeuron_2.png}
		\end{subfigure}
		\captionsetup{justification=centering}
		\caption{A single neuron}
		\label{fig:single neuron}
	\end{figure}
	
	The above neuron composed of multiple elements, which have different features and play various role in the neuron networks. Besides the inputs and outputs, the weights and activation function are the most important parts, which decide the basic function of a neuron.
	\begin{itemize}
		\item Connections and weights: The network consists of connections, each connection transferring the output of a neuron $i$ to the input of a neuron $j$. In this sense $i$ is the predecessor of $j$ and $j$ is the successor of $i$, each connection is assigned a weight $W_{ij}$
		\item Activation function: the activation function of a node defines the output of that node given an input or set of inputs. A standard computer chip circuit can be seen as a digital network of activation functions that can be "On" (1) or "Off" (0), depending on input. This is similar to the behavior of the linear perceptron in the neural networks. However, it is the nonlinear activation function that allows such networks to compute nontrivial problems using only a small number of nodes. In artificial neural networks this function is also called the transfer function.
	\end{itemize}
	\item \textbf{Feedforward Neural Network:}\\
	There are many classes of neural networks and these classes also have sub-classes, e.g. feedforward, lateral and feedback connections. Here the most used ones will be listed - Feedforward neural network.  An example of a feedforward neural networks is shown in Figure 4.5.\\
	\begin{figure}[h]
		\includegraphics[width=0.9\linewidth]{Figures/Section3_FeedforwardNN.png} 
		\captionsetup{justification=centering}
		\caption{A feedforward Multi-layer perceptron(MLP)}
		\label{fig:MLP}
	\end{figure}
	A feedforward neural network can consist of three types of layers:
	\begin{enumerate}
		\item \textbf{Input Layer:} No computation is done here within this layer, they just pass the information to the next layer (hidden layer most of the time).
		\item \textbf{Hidden Layer:} In Hidden layers is where intermediate processing or computation is done, they perform computations and then transfer the weights (signals or information) from the input layer to the following layer (another hidden layer or to the output layer). It is possible to have a neural network without a hidden layer, which is called Single layer Perceptron.
		\item \textbf{Output Layer:} Here an activation function can be used to maps to the desired output format (e.g. softmax for classification) .
	\end{enumerate}
	The feedforward neural network was the simplest type of artificial neural network devised, where connections between the units do not form a cycle. In this network, the information moves in only one direction, forward, from the input nodes, through the hidden nodes (if any) and to the output nodes. \\
	Two examples of feedforward networks are given below:
	\begin{enumerate}
		\item \textbf{Single Layer Perceptron - } This is the simplest feedforward neural network and does not contain any hidden layer. which means is only consists of a single layer of output nodes. This is said to be single because when we count the layers we do not include the input la
		\item \textbf{Multi Layer Perceptron - } A Multi Layer Perceptron has one or more hidden layers. 
	\end{enumerate}
Below only the Multi Layer Perceptrons will be discussed since they are more useful than Single Layer Perceptrons for practical applications.

\end{enumerate}


\subsection{Commonly used activation functions}
\subsection{Gradient Descent for Neural Networks}
\subsection{Forward Propagation and Back Propagation}
\subsection{title}
